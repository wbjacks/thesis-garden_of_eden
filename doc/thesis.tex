\documentclass{article}
    \title{Thesis Proposal: Garden of Eden}
    \author{William B. Jackson}
    \date{February 5, 2014}

    \newcommand{\tab}{\hspace*{2em}}
    \usepackage{fullpage}
    \usepackage{alltt}

\begin{document}
    \maketitle
    \section{(really bad) Abstract} % NOTE: this is really bad and needs to be re-written.
    \tab Garden of Eden (GoE) is an exercise in procedural generation of lifelike worlds. Using
parallelized computations, it randomly generates a forest scene of realistically shaped and
proportioned asymmetric trees on top of a basic topographical map. This map is then rendered in 3D,
with support for user navigation. The end result of this project is a sort of game: a static
rendering of a natural environment, exploring how humans find visual pleasure and meaning in virtual
environments. The passive interaction of the user is vital to this simulation, as it reflects how
one would observe a natural environment; ideally, the simulation would evoke similar reactions to
the environment GoE seeks to emulate, thus exploring the concept of a “natural”, the distinction
between real and virtual, and one's own sense of place.

    \section{Prior Art}

    \section{Terrain}
        \subsection{Midpoint Displacement}
\tab The Midpoint Displacement algorithm, also known as the Diamond-Square algorithm, is an easy
way to generate surface height maps for terrain. The algorithm is as follows:
    \begin{verbatim}
Loop while quadrilateral side is at least 1
    // Square step
    For each discrete square of given quadrilateral side length
        Get midpoint of square
        Calculate random offset as a random number in a specified range times a scale
        Midpoint becomes average of square's corners plus random offset

    // Diamond step
    For each discrete diamond defined by the square step's calculated midpoints
        Get midpoint of diamond
        Calculate random offset as a random number in a specified range times a scale
        Midpoint becomes average of diamond's corners plus random offset

    Decrease scale by a constant fraction
    Half quadrilateral side

    \end{verbatim}
    \tab The algorithm is works well to generate fractal surfaces, and allows for a decent amount
of control over the percieved "roughness" of the resultant surface in how one defines the scale
factor. However, it is fairly slow, executing in $O((n-1)^2)$ time. There are also well-documented
artifacts, notably for steeper surfaces.

        \subsection{Perlin Noise}

        \subsection{GoE Algorithm}

    \section{Trees}
        \subsection{L-Systems}
    \tab An L-System describes a recursive substitution system to build increasingly complex
strings. A system consists of an initial value or set of values, and a series of rules used to
recursively replace sections of the initial set. The set consists of productions, or operations
that, when combined with a rule, represent a certain kind of predictable growth. Productions can,
for example, represent a single branching structure, or trunk growth. As such, L-Systems are an
excellent way to represent the growth of trees, as they give a simple "seed" rules by which it can
branch in a biologically feasable way into a much larger system. As L-Systems consist of
self-replacement rules, they tend to form fractal systems, which are "self-similar" at different
levels of detail. In an L-System, this self-similarity at smaller levels of detail comes from the
recursion depth, which replaces productions at small levels with smaller versions of the designs
seen at a larger scale.


            \subsubsection{Productions}
    \tab A production defines a particular recursive structure. In combination with its replacement
rule, it acts similarly to a replacement function; a single production can be replaced by a set of
other productions, including itself, leading to interesting fractal shapes. In this project, a
production does not represent any physical shape, and is not drawn graphically; instead, it's 
replacement rule can contain graphical information, allowing a production to represent a specific
physical occurance. A production can, for example, represent a bijunction branching, or a bending
of a branch due to biological influences, or the thickening of the trunk due to age.

            \subsubsection{Parametric L-Systems}
    \tab A parametric L-System contains parametric productions, or, more simply, producitons that
take parameters. Production rules change to accomidate parameter logic, matching, for example, only
if a production has a "time" greater than 5. In this way, productions representing specific growth
actions can have significantly different geometry while still having the same basic structure. A
branch production, for example, might take an angle parameter, which would make larger branches
diverge less dramaticaly than twigs.

    \tab The below L-System can serve as an example, and is used as the L-System module's tester in
\verb|./test/lsys\_console.html|.
    % This is broken
    \begin{alltt}
    \(\omega\)  : B(2)A(4,4)
    \(p\sb{1}\) : A(x,y) : \(y \leq 3 \rightarrow\) A(x*2, x+y)
    \(p\sb{2}\) : A(x,y) : \(y > 3 \rightarrow\) B(x)A(x/y, 0)
    \(p\sb{3}\) : B(x)   : \(x < 1  \rightarrow\) C
    \(p\sb{4}\) : B(x)   : \(x \geq 1 \rightarrow\) B(x-1)   
    \end{alltt}
In this example, $\omega$ gives the initial productions with parametric values. The rules have
changed both to check the values of the parameters, and replace accordingly, and to modify the
parameters in the replacement step. In this way, the productions will have different parameters at
each level of recursion, reflecting changing growth at more depth.

            \subsubsection{Turtle Graphics}
    \tab As stated before, productions give no graphical information as to how a tree is drawn.
Thus, in addition to productions, an L-System must implement a system of graphical information. In
order to create branching structures, LOGO-style turtle graphics are used to draw on the HTML5
canvas. Invented as a simple graphics drawer for educational purposes, turtle graphics consist of a
set of commands that direct the movement of a "turtle", or a drawing point. In 3D space, three
dimensions of orientation are controlled by issuing commands with a specified radial change;
drawing occurs by moving the turtle forward. The turtle commands implemented are as follows:

\begin{center}
\begin{tabular}{|l|l|l|p{7cm}|}
    \hline
    Command & Symbol & Argument & Description \\ \hline \hline
    \_F & F & time & Moves turtle forward by specified time value (turtle rate is constant),
        drawing a line (a cylinder primitive) from its previous location to its new one. \\\hline
    \_f & f & time & Moves turtle forward by specified time value, drawing no line between
        positions. \\\hline
    \_pitch & \& & radians & Rotate turtle about its own x-axis by specified amount of
        radians. \\\hline
    \_yaw & + & radians & Rotate turtle about its own y-axis by specified amount of
        radians. \\\hline
    \_roll & / & radians & Rotate turtle about its own z-axis by specified amount of
        radians. \\\hline
    \_set & ! & width & Sets the drawn line (3D cylinder) to the specified width
        (diameter). \\\hline
    \_push & [ & none & Pushes current turtle position and orientation into a stack of previous
        states. \\\hline
    \_pop & ] & none & Pops current turtle position and orientation from a stack of previous
        states, resetting its current state to the new one. \\\hline

\end{tabular}
\end{center}
    \tab The above commands are included in production replacement rules, and basically define what
a production draws. In this way, a production might be used not just to define where a branch or
other such growth action occurs, but also to draw the branch upon its replacement with turtle
graphics commands. The amount of turtle graphics commands required to draw an appropriately
detailed tree is incredibly large; the exact "turtle string", or set of commands, used to draw one
tree is printed to the console in \verb|./test/tree.html|.

        \subsection{Biological Considerations}
    \tab As with most graphical projects, the generation of trees must strike a balance between
emulating actual biological methods and using efficient computational design. In this section, I
will discuss some of the biological phenomenon modeled in the creation of trees, how they were
modeled, and the effect that they have on tree growth. 

            \subsubsection{Discussion of Assumptions}
    \tab Some assumptions were made to simplify and speed up the generation of tree structures:
\begin{enumerate}
    \item Straight branches
    \item Outside forces kept to a minimum
    \item All trees of similar age, growing at a similar speed
    \item Clumping is not affected by light shielding
\end{enumerate}

            \subsubsection{Bijunctions} % Consider mentioning bronchial systems?
            \subsubsection{Tropism}
            \subsubsection{Growth Density}
            \subsubsection{Elevation and Gradation}

        \subsection{Randomization}
            % I might need to play with this a little more
            % Parameter variation
            % Probabilistic l-systems

    \section{Implementation}
        \subsection{Top Level}

        \subsection{Terrain Implementation}
            \subsection{terrain\_gen.js}
        \subsection{Tree Implementation}
    \tab The creation of a tree begins by creating a rule set for the L-System. A few of these are
defined in \verb|./js/app/lsys_rules.js|, which can be used as an example for users wishing to
create their own L-Systems. This rule set is used to create an L-System object, which will be used
to generate the tree's geometry. A turtle must also be constructed, and must be given a set of
three.js objects in order to draw to the HTML5 canvas. Then, by calling the L-System object's
\verb|build()| method, the L-System will recursively construct a tree, storing it in its own
\verb|property| public variable. This is then passed into the turtle object's \verb|run()| method,
which will draw the constructed L-System as a single hierarchical object onto the HTML5 canvas.

            \subsubsection{turtle\_graphics.js}
    \tab The turtle graphics module consists of a turtle object, \verb|Turtle|, a set of turtle
commands to control drawing, and a \verb|run| method, which, given a list of turtle commands,
executes them in order from the current position.
\begin{itemize}
    \item \verb|Turtle(scene, material, radius)|: Turtle object constructor. Creates the turtle
drawer, initializing it at the world origin oriented in the $+z$ direction. Turtle properties
are as follows:
    \begin{enumerate}
        \item \verb|rate|: Defaults to 1. Rate at which turtle travels when moving forward,
    multiplied by the forward command's \verb|time| parameter to get the total distance traveled.
        \item \verb|width|: "Width" of the drawn line. Defaults to 0.25, can be passed in as an
    argument. Actual value is the radius of the cylinder primitive created by the turtle.
        \item \verb|scene|: Three.js scene. Must be passed in. Graphics drawn in \verb|run()| are
    added to this as a single, hierarchical object.
        \item \verb|material|: Three.js material. Must be passed in. Defines the material of the
    drawn cylinder lines.
    \end{enumerate}

    \item Turtle commands: The turtle commands (\verb|_F()|, \verb|_f()|, \verb|_pitch()|,
\verb|_yaw()|, \verb|_roll()|, \verb|_push()|, \verb|_pop()|, and \verb|_set()|) are all methods
that specify commands to some turtle object. To use, they are attached to a \verb|Turtle.Action|
object along with the value of their argument, and then passed as a list as the argument to
\verb|run()|. Each one specifies a action relative to the turtle drawer's current position and
orientation. Details of the actions are given in table 1. In general, the rotation actions work by
multiplying a specific rotation matrix to the turtle object's internal orientation matrix. The use
of a rotation matrix allows for relative transforms, as opposed to Euler coordinates, which require
specific ordering of rotations which can lead to errors. The forward movement controls drawing; it
does this by creating a cylinder of the current width and given distance (the actual given is a
time parameter, which is multiplied by the turtle's \verb|rate| to get the cylinder's length) and
moving it such that it begins at the turtle's starting position and is oriented in the same
direction of the turtle. Then, the turtle's position is incremented by the given distance.
\verb|_push()| and \verb|_pop()| save the turtle's state in a basic object and store or retrieve it
from an internal stack.

    \item \verb|run()|: The run commands executes a given turtle string. Looping through the string
in order, it run's the function's built-in \verb|call()| function, specifying \verb|this| as the
calling object and \verb|Turtle.Action.args| as the arguments. This binds the current turtle object
to the action, allowing the turtle command to affect the turtle object's internal variables.

\end{itemize}
            \subsubsection{lsys.js}
    \tab \verb|lsys.js| gives the object used to describe and create an L-System of arbitrary
productions and rule set. Relevant object, including \verb|LSystem.Production| and
\verb|LSystem.RuleSet|, are defined, and the engine to run recursion to a specific depth is
included in a method of the base \verb|LSystem| object. The L-System requires both a rule set and
an initial value to run, both of which are described below.

\begin{itemize}
    \item \verb|LSystem|: Object containing the L-System and it's rule set. The only argument is
\verb|rule_table|, the rule set for the L-System.
    \item \verb|LSystem.Production|: Object containing production information. Productions, with
the added information of their rule sets, represent a specific type of growth, but are not actually
drawn by the turtle graphics wrapper. Production arguments are as follows:
    \begin{enumerate}
        \item \verb|id|: ID of the production, similar to function name. For example, in the
production A(1,4), the id is "A".
        \item \verb|args|: List of argument values. This is only assigned for initial values; the
\verb|inject| function is used to dynamically assign these values, as the resultant value is
usually some function of the previous value, defined in the rule set.
        \item \verb|inject_args|: Function used to dynamically assign arguments from within the
rule set. Arguments after the replacement step are usually a function of their previous values;
the width of branches, for example, might decrease by a constant scale at each level of branching.
The function must take \verb|args|, the previous production's arguments, and \verb|consts|, a
dictionary of constants defined in the rule set. It must then assign \verb|this.args| to a list of
the argument values.

    \end{enumerate}
    \item \verb|LSystem.RuleSet| and \verb|LSystem.Rule|: \verb|RuleSet| and \verb|Rule| are used
to define the logic of the L-System. \verb|RuleSet| describes the entire rule system, whereas Rule
defines a single rule for a specific production and condition. The arguments for the \verb|RuleSet|
constructor are as follows:
    \begin{enumerate}
        \item \verb|consts|: Dictionary of defined constants, such as width reduction ratios, used
    in calculating the argument values at replacement.
        \item \verb|initial|: Initial production list values for the system.
        \item \verb|rules|: List of rule objects for this system.
See \verb|lsys_rule.js| and its below explanation for more details on \verb|LSystem.RuleSet|
construction.

    \end{enumerate}
The arguments for the \verb|Rule| constructor are as follows:
    \begin{enumerate}
        \item \verb|id|: ID corresponding to the production this rule affects.
        \item \verb|condition|: Condition function to the parametric term. Function takes as
    argument the production object and returns a boolean. Allows the rule to match on the
    value of the production's parameter. 
        \item \verb|output|: List of production and turtle action objects to replace the matched
    production with upon recursion. 

    \end{enumerate}
More details about the construction of a \verb|RuleSet| can be seen in the documentation for
\verb|lsys_rule.js|.

    \item \verb|build()|: Launcher function for the recursive L-System construction. Calls internal
recursion function on each element in the initial system; thus, it requires that the \verb|system|
variable is set to a list of initial values. The internal recursion runs for a single object in the
system list, matching it to the rule set and replacing it with the appropriate output string. It
then recursively calls itself on each element of the output array, to a specified \verb|MAX_DEPTH|
value. Optional argument \verb|debug| is a boolean that toggles printing of the constructed system
after construction.

    \item \verb|checkRule()|: Function that checks a given element of the system against the
system's rule set, returning the appropriate output. Does a linear search through the list of
rules, halting when a match occurs. Matches require that both the rule id and the production id
match, as well as the rule's \verb|condition| function returning a \verb|true|. The rule's output
is then cloned to avoid problems with Javascript's singleton objects affecting subsequent
production values. The arguments are then injected using the production's provided
\verb|inject_args| function, which allows dynamic evaluation of the arguments based on the
previous production. The output is then returned. If no match is found, the initial production is
returned unchanged.

    \item \verb|printSystem()|: Used for debugging. Pretty prints the current value of the
L-System's \verb|system| variable, in form \verb|id(args...)|. Used when the \verb|debug| flag is
turned on in \verb|build()|.

\end{itemize}

            \subsubsection{lsys\_rule.js}
    \tab \verb|lsys_rule.js| provides a convenient package of L-System rules used in \emph{The
Algorithmic Beauty of Plants}. The rules found in it are used to create the trees in the final
Garden of Eden project, but can also serve as examples to users wishing to generate their own
L-Systems. The structure for a rule set is given in the above description of
\verb|LSystem.RuleSet|.
    \begin{itemize}
        \item \verb|HondaTree|: Creates an object that inherits from \verb|LSystem.RuleSet|.
Description of a Honda L-System comes from \emph{SOURCE WEE}. The tree is of a constant height, but
branches with increasing detail at higher levels of recursion. Defined constants allow for a
decent amount of variation on what is otherwise a visually highly symmetrical tree.

        \item \verb|TernaryTree|: Creates an object that inherits from \verb|LSystem.RuleSet|.
Description of a Ternary L-System comes from \emph{SOURCE WEE}. By using L-System rules that match
to turtle commands, this tree actually grows both larger and more detailed at higher levels of
recursion, which more closely resembles the growth of a tree. There is only one branching rule,
which erroneously uses a trijunction at each branching node. The generated tree is visually quite
complex, but has some symmetrical artefacts visible from specific angles. Varying constants allows
for a large amount of variation on this tree structure.

    \end{itemize}

    \section{Runtime}
        \subsection{LList Algorithm}
            \subsubsection{Profile}
            \subsubsection{Asymptotic Behavior}
        \subsection{Tree Search Algorithm}
            \subsubsection{Profile}
            \subsubsection{Asymptotic Behavior}
        \subsection{Three.js Considerations}
\end{document}
