\documentclass{article}
    \title{Thesis Proposal: Garden of Eden}
    \author{William B. Jackson}
    \date{February 5, 2014}

    \newcommand{\tab}{\hspace*{2em}}

    \usepackage[letterpaper]{geometry}

\begin{document}
    \maketitle
    \section{Abstract}
\tab Garden of Eden (GoE) is an exercise in procedural generation of lifelike
worlds. Using parallelized computations, it randomly generates a forest scene of
realistically shaped and proportioned asymmetric trees on top of a basic
topographical map. This map is then rendered in 3D, with support for user
navigation. The end result of this project is a sort of game: a static rendering
of a natural environment, exploring how humans find visual pleasure and meaning
in virtual environments. The passive interaction of the user is vital to this
simulation, as it reflects how one would observe a natural environment; ideally,
the simulation would evoke similar reactions to the environment GoE seeks to
emulate, thus exploring the concept of a “natural”, the distinction between real
and virtual, and one's own sense of place.

    \section{Problem}
\tab Asymmetric trees are particularly interesting to render accurately. As
opposed to symmetric trees, which can be probabilistically branched, asymmetric
trees must more closely copy the actual growth process found in nature. Branches
do not branch in mirroring directions, though they often sprout from a common
fork. Branches tend to grow away from each other, though factors such as
location of light, wind, and weather have their effects. Thus, the potential for
parallel computing is large, as trees in the forest affect the growth of their
neighbors. Terrain modeling is another problem, though one for which there is a
fairly significant amount of research. Fractal techniques will be evaluated for
their use in the creation of topographical maps, though other algorithms will
have to be introduced if natural structures, such as bodies of water, will be
generated.

    \section{Topics of Research}
\begin{enumerate}
    \item \emph{Fractal landscapes}-Used to generate realistic terrain maps.
Includes Fractional Brownian Motion and mutifractal generation techniques.
    \item \emph{Computational optimization}- Generation must be reasonably
quick, and rendering must be fast enough for dynamic interaction.
    \item \emph{Feature generation}- Natural features such as water might
require special rules.
    \item \emph{Asymmetric tree generation}- The focus of the project will be in
rendering geometrically believable trees. To this end, generation techniques
will be examined in tandem to biological and physical growth rules.
    \item \emph{Artistic Theory}- The metric of success with this project is
ultimately the user experience: as I am trying to evoke emotional response in
the user rather than attempting to generate any sort of mathematically or
statistically verifiable response. As such, I will do some research in similar
artistic projects of various media. There is a lot of literature on the concepts
place and reality. I will also examine Chinese landscape paintings, which
similarly explored one's perception of the natural world.
\end{enumerate}

    \section{Technologies}
\begin{enumerate}
    \item \emph{Graphics}- My goal is to display this in a browser, as it makes
distribution of the software far simper. I will likely use three.js
(www.threejs.org) to render GoE. Three.js uses WebGL, the browser version of
OpenGL, to achieve 3D, hardware-accelerated graphics. While not as powerful as a
native OpenGL app due to the overhead of running on a browser, it is
nevertheless a powerful, and simple, option. Processing.js, which is also
optimized with WebGL, is being considered as a back-up option.

    \item \emph{Middleware}- I want to take advantage of native processing speed
for the generation of the forests, so I need some sort of message passing
protocol to inform the graphics end what to render. My plan here is to render
from a JSON serialization of the tree data structures, which is then sent by
HTTP to a node on the graphics application.

    \item \emph{Backend}- Native code, ideally using a parallel protocol. Hadoop
and OpenGL are considered for their distributed parallel processing, but GPU
processing on a single host is a tempting option for ease of deployment.

    \item \emph{Alternate option}- it might be possible to get appropriate
render speeds by backgrounding the generation process and running the entire
application as a web app. This would have the benefit of a much more streamlined
deployment, as the user would only have to access the online canvas. Hadoop
could be utilized in this solution for parallelization, but it would demand an
increased load on the host server.

    \item \emph{Second alternate option}- have the generation run as a
Javascript program that would be automatically downloaded by the browser as the
user accessed the web app, and would subsequently send an AJAX post to the
server, in accordance to the proposed middleware. This would be a simpler and
easier to distribute version of the initially proposed stack, as all client-side
files would be automatically managed by the browser, but I am not sure that
Javascript would provide the computation speed I desire for the scene
generation.

\end{enumerate}

    \section{Prior Art}
        \subsection{Terragen 3 (http://planetside.co.uk/products/terragen3)}
\tab Terragen 3 is a proprietary tool for rendering and animating realistic
natural environments. The suite is able to render photorealistic images and
animations and comes in both a free and commercial version. Terragen is
appropriate for generating static images and animations, but is not appropriate
for dynamic interaction. It also allows the user a high level of control over 
the generation, as the software is intended for designers and animators.

        \subsection{Minecraft (https://minecraft.net/)}
\tab Popular video game that includes significant and believable, though
cartoonish, terrain generation, including biomes, elevation, and natural
“features” like caves and chasms. Designed as a sandbox-style game that allows
the user an almost infinite amount of control over the world in the form of
creating and destroying world “blocks”. While the methods of procedural
generation used in Minecraft will be examined, GoE is different in that there is
no control over the appearance of the generated world. Additionally, there will
be a greater focus placed on generating realistic trees.

        \subsection{Proteus (http://www.visitproteus.com/)}
\tab Another indie game that features a randomly-generated island. The feel of
this game is similar to GoE in that the sole goal is exploration. Proteus,
however, has a whimsically-styled world, whereas GoE will seek to evoke actual
natural scenes.

\end{document}
